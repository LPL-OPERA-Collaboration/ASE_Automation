\documentclass[11pt, a4paper]{article}

% --- Essential Packages ---
\usepackage[utf8]{inputenc}
\usepackage[T1]{fontenc}
\usepackage{geometry}
\geometry{margin=2.5cm}
\usepackage{hyperref}
\usepackage{listings}
\usepackage{xcolor}

% --- Simple Code Style ---
\lstset{
    basicstyle=\ttfamily\small,
    frame=single,
    breaklines=true,
    backgroundcolor=\color{gray!10}
}

% --- Title ---
\title{Angle-Resolved Spectroscopy: Everyday User Manual}
\author{OPAL Lab}
\date{\today}

\begin{document}

\maketitle
\tableofcontents
\newpage

% =============================================================================
% PREREQUISITES
% =============================================================================
\section*{Prerequisites}
Before starting, ensure you have:
\begin{enumerate}
    \item \textbf{Calibration File:} The \texttt{wheel\_calibration.csv} file in the \texttt{gentec data} folder.
    \item \textbf{Absorption Data:} The UV-Vis absorption spectrum of your sample.
    \item \textbf{Hardware:} The laser should be warm and the fiber aligned.
\end{enumerate}

\hrule
\vspace{1em}

% =============================================================================
% PART A
% =============================================================================
\section{Part A: Data Acquisition}

\subsection{1. Power-Up Sequence}
Turn on the equipment in this exact order:
\begin{enumerate}
    \item \textbf{Horiba iHR550} (Monochromator Switch)
    \item \textbf{Horiba Synapse} (Plug in Camera Power)
    \item \textbf{SDRIVE-500} (Shutter Switch)
    \item \textbf{QC Sapphire} (Pulser Switch)
    \item \textbf{N2 Laser} (Turn Key)
    \item \textbf{Thorlabs Motor} (Powered via PC)
\end{enumerate}

\subsection{2. Manual Alignment (Critical Step)}
Before automation, align the system manually.

\begin{enumerate}
    \item Open \textbf{LabSpec 6}, \textbf{ELLO} (Motor), and \textbf{QC\_9200} (Pulser).
    \item Move the sample to a high-signal angle (e.g., 280 degrees) using ELLO.
    \item Start \textbf{Real-Time Display} in LabSpec.
    \item Align the detector to maximize signal.
    \item \textbf{Check for Saturation:} Ensure the signal is below 60,000 counts. 
    \begin{itemize}
        \item If saturated: Add an ND filter or move the detector further away.
    \end{itemize}
\end{enumerate}

\textbf{IMPORTANT:} You \textbf{MUST CLOSE} LabSpec, ELLO, and QC\_9200 before proceeding. If they remain open, the automation script will fail.

\subsection{3. Configuration}
Open \texttt{aquisition\_config.py} and edit Section 2:

\begin{itemize}
    \item Set the save folder:
\begin{lstlisting}[language=Python]
BASE_SAVE_DIRECTORY = r"C:\Users\Equipe_OPAL\Desktop\Kaya\data"
\end{lstlisting}
    
    \item Set the scan angles:
\begin{lstlisting}[language=Python]
START_ANGLE = 85.0
END_ANGLE = 280.0
\end{lstlisting}

    \item Set exposure times (Longest to Shortest):
\begin{lstlisting}[language=Python]
INTEGRATION_TIME_PRESETS_S = [4.0, 0.1]
\end{lstlisting}
\end{itemize}

\subsection{4. Run Experiment}
Run \texttt{main\_measurement.py}.
\begin{itemize}
    \item The script creates a new folder (e.g., \texttt{Measurement\_1}).
    \item If the signal saturates, the script will pause. Choose \textbf{[S]top}, fix the saturation (filter/distance), and restart.
\end{itemize}

\newpage

% =============================================================================
% PART B
% =============================================================================
\section{Part B: Data Analysis}

\subsection{1. File Preparation}
Go to your new folder (e.g., \texttt{Measurement\_1}) and paste these two files inside:
\begin{itemize}
    \item The \textbf{Calibration .csv} (from \texttt{gentec data}).
    \item The \textbf{Absorption .txt/.csv} (of your sample).
\end{itemize}

\subsection{2. Configuration}
Open \texttt{analysis\_config.py}:
\begin{enumerate}
    \item Update the folder path:
\begin{lstlisting}[language=Python]
BASE_DIR = r"C:\Users\Equipe_OPAL\Desktop\Kaya\data\20251127_Measurement_1"
\end{lstlisting}
    \item Input the reference energy measured today:
\begin{lstlisting}[language=Python]
RAW_ENERGY_READ_NJ = 24  # Energy in nJ
TODAYS_OD = OD3          # Filter used (OD1, OD3, or NO_OD)
\end{lstlisting}
\end{enumerate}

\subsection{3. Run the Pipeline}
Run the scripts in order:

\begin{enumerate}
    \item \textbf{Run \texttt{step1\_energy\_calc.py}:} 
    \begin{itemize}
        \item Calculates physics (Fluence).
        \item Check the "Energy Profile" plot for smoothness.
    \end{itemize}
    
    \item \textbf{Run \texttt{step2\_signal\_processing.py}:}
    \begin{itemize}
        \item Smooths the spectra.
    \end{itemize}
    
    \item \textbf{Run \texttt{step3\_spectrum\_analysis.py}:}
    \begin{itemize}
        \item Calculates Intensity and FWHM.
        \item Outputs \texttt{FINAL\_RESULTS.csv} and the ASE Threshold plot.
    \end{itemize}
\end{enumerate}

\section*{Troubleshooting}
\begin{itemize}
    \item \textbf{Connection Error:} You probably left LabSpec open. Close it.
    \item \textbf{Saturation Warning:} Signal is too strong. Add an ND filter.
    \item \textbf{Calibration File Missing:} You forgot to copy the .csv file into the measurement folder.
    \item \textbf{Analysis Crash:} Check the \texttt{BASE\_DIR} path in config.
\end{itemize}

\end{document}